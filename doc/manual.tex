\documentclass[a4paper]{article}

\begin{document}

\section{Options}
The options should be put before all expression in the input file.

All options start with ``\%'' or ``\#'' or ``\$'', followed by a word indicate the option name, and then a number or string which is the option value. 

Examples:
\begin{itemize}
\item \textbf{\%number} 1
\item \textbf{\%name} feiwoa
\item \textbf{\%filename} ``exfile 1''
\end{itemize}

As demonstrated in the third example, if there is space in the option value, the quotes should be added.

\subsection{Options that control run procedure}
\begin{itemize}
\item \textbf{\%strategy}
This option control the main run procedure of WPO.

Supported values are the following:
\begin{itemize}
  \item \textbf{kcm}: using kcm method.
\item \textbf{fastrun}: Do simple optimization, which is fast but results are not so good. Recommend for extreme heavy input expressions.
\end{itemize}

\end{itemize}

\subsection{Options that control output style}
\begin{itemize}
\item \textbf{\%tmp\_style}
This option controls the output style for temporary variables.

There are two possible values for this option:
\begin{itemize}
\item \textbf{pre}: the declaration of temporary variables will be put before the calculation.
\item \textbf{in}: the declaration of temporary variable will be put when it first appears(recommend, default).
\end{itemize}


\item \textbf{\%func\_style}
This option controls the output style for temporary variables used for functions.

Like \textbf{\%tmp\_style}, ``pre'' and ``in'' are supported.

\item \textbf{\%var\_style}
This option controls the declaration for the target variable, i.e. the variable before '=' in the input file.

Like \textbf{\%tmp\_style}, ``pre'' and ``in'' are supported. Besides these two options, ``no'' is supported, which means that in output file, these variable will not be declared by WPO.

\item \textbf{\%tmp\_prefix}
This is the prefix of the temporarary variable name.
\item \textbf{\%tmp\_suffix}
This is the suffix of the temporary variable name.

\item \textbf{\%type}
This is the type used in the declaration of temporary variables (and other variables).

\item \textbf{\%line\_prefix}
This is the prefix in each output line.

\item \textbf{\%line\_suffix}
This is the suffix in each output line. Default value for it is ``;'', corresponding to the C programming language.

\item \textbf{\%func\_prefix}
This is the prefix of temporary variables for functions.

\end{itemize}

\end{document}
